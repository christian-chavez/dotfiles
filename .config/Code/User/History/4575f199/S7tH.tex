\def\thetitle{Homework 2}
\input{title.tex}


\begin{questions}
\question Determine which of the following binary operations are associative.
\begin{enumerate}[label=(\alph*)]
    \item the operation \(\star\) on \(\Z\) defined by \(a\star b  = a-b\)
    \item the operation \(\star\) on \(\R\) defined by \(a\star b  = a+b+ab\)
    \item the operation \(\star\) on \(\Q\) defined by \(a\star b  = \dfrac{a+b}{5}\)
    \item the operation \(\star\) on \(\Z\times\Z\) defined by \((a , b) \star (c,d)  = (ad+bc, bd)\)
    \item the operation \(\star\) on \(\Q\backslash\left\{ 0 \right\}\) defined by \(a\star b = \dfrac{a}{b} \)
\end{enumerate}

\question 
Prove that addition of residue classes in \(\Z / n \Z\) is associative. (Assume it is well defined.)


\question
Determine which of the following sets are groups under addition:
\begin{enumerate}[label=(\alph*)]
    \item the set of rational numbers (including \(0=0 / 1\)) in lowest terms whose denominators are odd
    \item the set of rational numbers (including \(0=0 / 1\)) in lowest terms whose denominators are even
    \item the set of rational numbers of absolute value \(<1\)
    \item the set of rational numbers of absolute value \(\geq 1\) together with 0
    \item the set of rational numbers with denominators equal to 1 or 2
    \item the set of rational numbers with denominators equal to 1, 2 or 3 
\end{enumerate}



\question
Let \(G=\left\{z \in \mathbb{C} \mid z^n=1 \text{ for some }n \in \mathbb{Z}^{+}\right\}\).
\begin{enumerate}[label=(\alph*)]
    \item Prove that \(G\) is a group under multiplication (called the group of \textit{roots of unity} in \(\mathbb{C}\)).
    \item Prove that \(G\) is not a group under addition.
\end{enumerate}


\question
Let \(G=\{a+b \sqrt{2} \in \mathbb{R} \mid a, b \in \mathbb{Q}\}\).
\begin{enumerate}[label=(\alph*)]
    \item Prove that \(G\) is a group under addition.
    \item Prove that the nonzero elements of \(G\) are a group under multiplication. (``Rationalize the denominators'' to find multiplicative inverses.)
\end{enumerate}


\question
Find the orders of each element of the additive group \(\mathbb{Z} / 12 \mathbb{Z}\).

\question
Find the orders of the following elements of the multiplicative group \((\mathbb{Z} / 12 \mathbb{Z})^\times\):
\[\overline{1}, \overline{-1}, \overline{5}, \overline{7}, \overline{-7}, \overline{13}.\]

\question
Find the orders of the following elements of the additive group \(\mathbb{Z} / 36 \mathbb{Z}\):
\[ \overline{1}, \overline{2}, \overline{6}, \overline{9}, \overline{10}, \overline{12}, \overline{-1},\overline{-10},\overline{-18}.\]



\question
Let \(x\) be an element of \(G\). Prove that \(x^2=1\) if and only if \(|x|\) is either 1 or 2.

\question
Let \(x\) be an element of \(G\). Prove that if \(|x|=n\) for some positive integer \(n\) then \(x^{-1}=x^{n-1}\).

\question
Let \(x\) and \(y\) be elements of \(G\). Prove that \(x y=y x\) if and only if \(y^{-1} x y=x\) if and only if \(x^{-1} y^{-1} x y=1\).


\question
Let \(x \in G\) and let \(a, b \in \mathbb{Z}^{+}\).
\begin{enumerate}[label=(\alph*)]
    \item Prove that \(x^{a+b}=x^a x^b \) and \(\left(x^a\right)^b=x^{a b}\).
    \item Prove that \(\left(x^a\right)^{-1}=x^{-a}\).
    \item Establish part (a) for arbitrary integers \(a\) and \(b\) (positive, negative or zero).
\end{enumerate}

\question
For \(x\) an element in \(G\) show that \(x\) and \(x^{-1}\) have the same order.


\question 
If \(x\) and \(g\) are elements of the group \(G\), prove that \(|x|=\left|g^{-1} x g\right|\). Deduce that \(|a b|=|b a|\) for all \(a, b \in G\).

\question
 Prove that if \(x^2=1\) for all \(x \in G\), then \(G\) is abelian.

\question
Assume \({H}\) is a nonempty subset of \((G, \star)\) which is closed under the binary operation on \(G\) and is closed under inverses, i.e., for all \(h\) and \(k\) elements of \(H\) it holds    \(hk,h^{-1} \in H\). Prove that \(H\) is a group under the operation \(\star\) restricted to \(H\) (such a subset \(H\) is called a subgroup of \(G\) ).

\question
Prove that if \(x\) is an element of the group \(G\) then \(\left\{x^n \mid n \in \mathbb{Z}\right\}\) is a subgroup (cf. the preceding exercise) of \(G\) (called the cyclic subgroup of \(G\) generated by \(x\)).

\question
Compute the order of each of the elements in  (a) \(D_6\), (b) \(D_8\), and (c) \(D_{10}\).



\question
Let \(\sigma\) be the permutation
\[
1 \mapsto 3 \quad 2 \mapsto 4 \quad 3 \mapsto 5 \quad 4 \mapsto 2 \quad 5 \mapsto 1
\]
and let \(\tau\) be the permutation
\[
1 \mapsto 5 \quad 2 \mapsto 3 \quad 3 \mapsto 2 \quad 4 \mapsto 4 \quad 5 \mapsto 1 .
\]

Find the cycle decompositions of each of the following permutations: \(\sigma, \tau, \sigma^2, \sigma \tau, \tau \sigma\), and \(\tau^2 \sigma\).


% Quedo en la página 4 del PDF


\question

Compute the order of each of the elements in the following in (a) \(S_3\) and (b) \(S_4\).

\question
Find the order of (1 12 8 10 4)(2 13)(5 11 7)(6 9).


\question
Write out the cycle decomposition of each element of order 4 in \(S_4\).


\question
\begin{enumerate}[label=(\alph*)]
    \item Let \(\sigma\) be the 12-cycle (1 2 3 4 5 6 7 8 9 10 11 12). For which positive integers \(i\) is \(\sigma^i\) also a 12-cycle?
    \item Let \(\tau\) be the 8-cycle (1 2 3 4 5 6 7 8). For which positive integers \(i\) is \(\tau^i\) also an 8-cycle?
    \item Let \(w\) be the 14-cycle (1 2 3 4 5 6 7 8 9 10 11 12 13 14). For which positive integers \(i\) is \(\omega^i\) also a 14-cycle?
\end{enumerate}

\question\label{q.10.sec.1.3}
Prove that if \(\sigma\) is the \(m\)-cycle \(\left(a_1 a_2 \ldots a_m\right)\), then for all \(i \in\{1,2, \ldots, m\}\), it holds \(\sigma^i\left(a_k\right)=a_{k+i}\), where \(k+i\) is replaced by its least residue mod\;\(m\) when \(k+i>m\). Deduce that \(|\sigma|=m\).

\question
Let \(\sigma\) be the \(m\)-cycle \((1\; 2\; 3\; \cdots\; m)\). Show that \(\sigma^i\) is also an \(m\)-cycle if and only if \(i\) is relatively prime to \(m\).


\question
Let \(p\) be a prime. Show that an element has order \(p\) in \(S_n\) if and only if its cycle decomposition is a product of commuting \(p\)-cycles. Show by an explicit example that this need not be the case if \(p\) is not prime.

\question
Prove that the order of an element in \(S_n\) equals the least common multiple of the lengths of the cycles in its cycle decomposition. (Hint: use problem \ref{q.10.sec.1.3}.)


\question
Prove that \(\abs{GL_2(\mathbb{F}_2)} = 6\).




\question
Write out all the elements of \(G L_2\left(\mathrm{~F}_2\right)\) and compute the order of each element.

\question
3. Show that \(G L_2\left(F_2\right)\) is non-abelian.

\question
4. Show that if \(n\) is not prime then \(\mathbb{Z} / n \mathbb{Z}\) is not a field.

\question
5. Show that \(G L_n(F)\) is a finite group if and only if \(F\) has a finite number of elements.

\question
6. If \(|F|=q\) is finite prove that \(\left|G L_n(F)\right|<q^{n^2}\).

\question
8. Show that \(G L_n(F)\) is non-abelian for any \(n \geq 2\) and any \(F\).

\hfill\null\par
\hspace*{-1cm}\fbox{\parbox{1.05\linewidth}{
The next exercise introduces the Heisenberg group over the field \(F\) and develops some of its basic properties. When \(F=\mathbb{R}\) this group plays an important role in quantum mechanics and signal theory by giving a group theoretic interpretation (due to H. Weyl) of Heisenberg's Uncertainty Principle. Note also that the Heisenberg group may be defined more generally for example, with entries in \(\mathbf{Z}\).
}}

\question
Let \(H(F)=\left\{\left.\left(\begin{array}{lll}1 & a & b \\ 0 & 1 & c \\ 0 & 0 & 1\end{array}\right) \;\right|\; a, b, c \in F\right\}\) --- called the Heisenberg group over \(F\).
Let \(X=\left(\begin{array}{lll}1 & a & b \\ 0 & 1 & c \\ 0 & 0 & 1\end{array}\right)\) and \(Y=\left(\begin{array}{lll}1 & d & e \\ 0 & 1 & f \\ 0 & 0 & 1\end{array}\right)\) be elements of \(H(F)\).

Compute the matrix product \(X Y\) and deduce that \(H(F)\) is closed under matrix multiplication. Exhibit explicit matrices such that \(X Y \neq Y X\) (so that \(H(F)\) is always non-abelian).



\question

Let \(G\) and \(H\) be groups.  Let \(\varphi: {G} \rightarrow {H}\) be a homomorphism.
\begin{enumerate}[label=(\alph*)]
    \item Prove that \(\varphi\left(x^n\right)=\varphi(x)^n\) for all \(n \in \mathbb{Z}^{+}\).
    \item  Do part (a) for \(n=-1\) and deduce that \(\varphi\left(x^n\right)=\varphi(x)^n\) for all \(n \in \mathbb{Z}\).
\end{enumerate}





\question

Let \(G\) and \(H\) be groups. If \(\varphi: G \rightarrow H\) is an isomorphism, prove that \(|\varphi(x)|=|x|\) for all \(x \in G\). Deduce that any two isomorphic groups have the same number of elements of order \(n\) for each \(n \in \mathbb{Z}^{+}\). Is the result true if \(\varphi\) is only assumed to be a homomorphism?

\question
Let \(G\) and \(H\) be groups. If \(\varphi: G \rightarrow H\) is an isomorphism, prove that \(G\) is abelian if and only if \(H\) is abelian. If \(\varphi: G \rightarrow H\) is a homomorphism, what additional conditions on \(\varphi\) (if any) are sufficient to ensure that if \(G\) is abelian, then so is \(H\)?

\question
Prove that \(D_{24}\) and \(S_4\) are not isomorphic.



\question
Let \(A\) and \(B\) be groups. Prove that \(A \times B \cong B \times A\).


\question
Let \(G\) and \(H\) be groups and let \(\varphi: G \rightarrow H\) be a homomorphism. 
Prove that the image of \(\varphi\)  is a subgroup of \(H\).
Prove that, if \(\varphi\) is injective, then \(G \cong \varphi(G)\).


\question
Let \(G\) and \(H\) be groups and let \(\varphi: G \rightarrow H\) be a homomorphism. Define the kernel of \(\varphi\) to be \(\left\{g \in G \mid \varphi(g)=1_H\right\}\) (so the kernel is the set of elements in \(G\) which map to the identity of \(H\), i.e., is the \textit{fiber} over the identity of \(H\)). Prove that the kernel of \(\varphi\) is a subgroup   of \(G\). Prove that \(\varphi\) is injective if and only if the kernel of \(\varphi\) is the identity subgroup of \(G\).

\question
Define a map \(\pi: \mathbb{R}^2 \rightarrow \mathbb{R}\) by \(\pi((x, y))=x\). Prove that \(\pi\) is a homomorphism and find the kernel of \(\pi\).

\question
Let \(G\) be any group. Prove that the map from \(G\) to itself defined by \(g \mapsto g^{-1}\) is a homomorphism if and only if \(G\) is abelian.


\question
Let \(G\) be any group. Prove that the map from \(G\) to itself defined by \(g \mapsto g^2\) is a homomorphism if and only if \(G\) is abelian.

\question
Let \(G=\left\{z \in \mathbb{C} \mid z^n=1\right.\) for some \(\left.n \in \mathbb{Z}^{+}\right\}\). Prove that for any fixed integer \(k>1\) the map from \(G\) to itself defined by \(z \mapsto z^k\) is a surjective homomorphism but is not an isomorphism.










\question










\question










\question



















































\end{questions}