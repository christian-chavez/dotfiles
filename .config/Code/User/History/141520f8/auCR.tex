\section{Ambientes de teoremas}

\lipsum[2]

\subsection{Teorema propio}
hello \lipsum[2]

\begin{theorem}[Buenas tardes]
    hola \[\int_a^b b = h\]
    jaja xd
\end{theorem}

\lipsum[1]

\section{Definiones}

\lipsum[1]

\begin{definition}
    Let \(f\) and \(g\) be two continuous maps from a topological space \(X\) into a topological space \(Y\). Suppose \(H \colon X\times I \to Y\) is a continuous map such that 
    \[
        \forall x\in X :\quad H(x,0) = f(x), \quad H(x,1) = g(x).
    \]
    We call \(H\) an \textbf{homotopy} from \(f\) to \(g\). We also say that \(f\) and \(g\) are \textbf{homotopic}, which will be denoted \(f\simeq g\). Maps homotopic to a constant map are said to be \textbf{null-homotopic}.
\end{definition}

\lipsum[1]

\begin{theorem}ola \[\int_a^b b = h\]
    jaja xd
\end{theorem}

\lipsum[1]

\begin{definition}
    Let \(f\) and \(g\) be two continuous maps from a topological space \(X\) into a topological space \(Y\). Suppose \(H \colon X\times I \to Y\) is a continuous map such that 
    \[
        \forall x\in X :\quad H(x,0) = f(x) \quad H(x,1) = g(x).
    \]
    We call \(H\) an \textbf{homotopy} from \(f\) to \(g\). We also say that \(f\) and \(g\) are \textbf{homotopic}, which will be denoted \(f\simeq g\). Maps homotopic to a constant map are said to be \textbf{null-homotopic}.

\end{definition}

\lipsum[1-2]