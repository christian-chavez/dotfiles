% Preambulo
\usepackage[T1]{fontenc}
\usepackage[utf8]{inputenc}
\usepackage[spanish,es-noindentfirst,es-tabla]{babel}
\usepackage{
    microtype,
    geometry,
    lmodern,
    lipsum,
    enumitem,
    amsmath,
    amsfonts,
    amssymb,
    amsthm,
    graphicx,
    float,
    booktabs,
    listings, % Para insertar código
    xcolor,
}

% Ambientes propios, teoremas, etc

\newtheoremstyle{uptheorem}{\baselineskip}{\baselineskip}{\slshape}{}{\bfseries}{.}{ }{} % nuevo estilo de teorema
\theoremstyle{uptheorem}    % activamos el estilo
\newtheorem{theorem}{Teorema}[section]  % creamos un tipo de teorema
\newtheorem{definition}[theorem]{Definición}

\newtheoremstyle{uptheorem}{\baselineskip}{\baselineskip}{\upshape}{}{\bfseries}{.}{ }{} % nuevo estilo de teorema (Problemas)
\theoremstyle{uptheorem}    % activamos el estilo
\newtheorem{example}[theorem]{Teorema}  % creamos un tipo de teorema
\newtheorem{problem}[theorem]{Problema}

% Estilo del código insertado
\renewcommand\lstlistingname{Listing}
\lstset{
    backgroundcolor=\color{gray!05},,
    basicstyle=\small\ttfamily,
    numbers=left,
    frame=tblr, 
    framerule=0.5pt, 
    breaklines=true,
    captionpos=t,
    showstringspaces=false,
    commentstyle=\color{gray},
    stringstyle=\color{orange},
    numberstyle=\color{gray}\small\ttfamily, 
    keywordstyle = \color{green!70!black}, % personalizar
    identifierstyle=\color{blue},   % mayoría de comandos
    numbersep=10pt, 
    emph={print,in,pop,push,not,set,list},
    emphstyle=\color{purple},
    tabsize=4,
    rulecolor=\color{black!30},
    escapeinside={\%*}{*)},
    extendedchars=true,
    xleftmargin=2.35em, 
    xrightmargin=0.32em,
    aboveskip=1em,
    belowskip=1em,
    lineskip=-0.05em,
    columns=fullflexible,
    framexleftmargin=2em, 
    framextopmargin=0.5ex, 
}

\usepackage{hyperref} % siempre va al final de los otros paquetes
\hypersetup{
    bookmarksnumbered,
    hidelinks,
    colorlinks,
    allcolors=blue,
    pdfproducer={Christian Chávez}
}


% Configuraciones
\setlength{\parskip}{\baselineskip}
% \setlength{\parindent}{5cm}
\raggedbottom
\geometry{
    margin=1in,
    % top=1cm,
    % left=2cm,
    % right=0.5cm,
    % bottom=8cm
}

% Datos
\title{Aprendiendo \LaTeX{} con Christian}
\author{Christian Chávez}
\date{2024-03-20}
