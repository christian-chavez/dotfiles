\usepackage[T1]{fontenc}
\usepackage[utf8]{inputenc}
\usepackage[english,]{babel}
\usepackage[super]{nth}
\usepackage{
    amsmath,
    amsfonts,
    amssymb,
    amsthm,
    imakeidx,
    graphicx,
    polynom,
    % subcaption,
    lipsum,
    lmodern,
    systeme,
    float,
    ifthen,
    mathrsfs,
    units,
    % enumerate,
    enumitem,
    multicol,
    geometry,
    mdframed,
    setspace,
    subcaption,
    anyfontsize,
    mathtools,
    % fancyhdr,
    booktabs,
    multirow,
    csquotes,
    microtype,
    bookmark,
    emptypage,
    blkarray,
    aliascnt,
    dsfont,
    comment,
    tabularx,
    % mathds,
    titlesec,
    cancel,
    currfile,
}
\usepackage[perpage]{footmisc}

\setenumerate{topsep=0pt}

\setlength{\parskip}{0.5\baselineskip}%

% \fontseries{b}\selectfont
% \renewcommand{\bfdefault}{b}

\titleformat{\chapter}
  {\normalfont\Huge}{\Huge\thechapter}{15pt}{\Huge}

% \titleformat{\section}
%   {\normalfont\large}{\large\thechapter}{15pt}{\large}

% \titleformat{\section}
%   {\normalfont\Huge}{\Huge\thechapter}{15pt}{\Huge}

% \onehalfspacing
\setstretch{1.15}
\geometry{
    margin=1in,
    headheight=50pt,
    headsep=12pt,
    % showframe
    % textwidth=5in,
}
% \decimalpoint
\raggedbottom
% \pagestyle{plain}
% \fancyhf{}
% % \fancyheadoffset{1.0mm}
% \lhead{\includegraphics[width=0.35\textwidth]{logoECMC.png}}
% \rhead{\includegraphics[width=0.25\textwidth]{logoYachay.png}}
% \fancyfoot[RO,LE]{\thepage}

\usepackage[delims=\lbrack\rbrack]{spalign}
\let\matrix=\spalignmat
\let\Amatrix=\spalignaugmat
\let\AAmatrix=\spalignaugmatn
\newcommand{\Dmatrix}[1]{\spaligndelims\vert\vert\spalignmat{#1}}

% \fancyhead[RE]{\scshape{\nouppercase{\leftmark}}}
% \fancyhead[LO]{\scshape{\nouppercase{\rightmark}}}
% \renewcommand{\headrulewidth}{0pt}
\graphicspath{{images/}}
\bookmarksetup{numbered}
\makeatletter
 \renewcommand\Hy@numberline[1]{#1. }
\makeatother
\usepackage{hyperref}
\hypersetup{%
    colorlinks = {true},
            linkcolor = {blue},
            urlcolor  = {blue},
            citecolor = {blue},
            anchorcolor = {blue},
    pdftitle    ={},
    pdfauthor   ={Christian Chávez},
    pdfsubject  ={Abstract Algebra},
    pdfcreator  ={LaTeX},
    pdfproducer ={Christian Chávez},
    % pdfkeywords={}{}{},
    % hidelinks,
    pdfpagemode={UseNone},
}
\newcolumntype{Y}{>{\centering\arraybackslash}X}
\renewcommand\tabularxcolumn[1]{m{#1}}
% \renewcommand\arraystretch{1.5}

\newcommand{\p}{\cdot}
% \newenvironment{solution}{\noindent\textbf{\textsl{Solution}.}}{\hfill$\square$}
% \newenvironment{remark}{\noindent\textbf{\textsl{Remark.}}}{}
% \theoremstyle{definition}
% \newtheorem{problem}{Problem}%[chapter]

% \setlength\parindent{0pt}

% Nuevos comandos
\let\originalleft\left
\let\originalright\right
% \renewcommand{\left}{\mathopen{}\mathclose\bgroup\originalleft}
% \renewcommand{\right}{\aftergroup\egroup\originalright}
\newcommand{\N}{\mathds{N}}
\newcommand{\Z}{\mathds{Z}}
\newcommand{\R}{\mathds{R}}
\newcommand{\Q}{\mathds{Q}}
\newcommand{\II}{\mathds{I}}
\newcommand{\K}{\mathds{K}}
\newcommand{\PP}{\mathds{P}}
\newcommand{\C}{\mathds{C}}
\newcommand{\T}{\mathscr{T}}
\newcommand{\I}{\mathcal{I}}
\newcommand{\E}{\mathscr{E}}
\newcommand{\B}{\mathscr{B}}
\newcommand{\F}{\mathscr{F}}
\newcommand{\U}{\mathscr{U}}
\newcommand{\LL}{\mathscr{L}}
\newcommand{\OO}{\mathscr{O}}
\newcommand{\Fam}{\mathscr{F}}
\newcommand{\Nei}{\mathscr{N}}
\newcommand{\Pt}{\mathscr{P}}
% \newcommand{\ES}{\text{\upshape\O}}
% \renewcommand{\ES}{\cancel{\mathrm{O}}}
\newcommand{\ES}{\varnothing}
\newcommand{\TopS}{\left( X, \T \right)}
\newcommand{\bcap}{\bigcap}
\newcommand{\bcup}{\bigcup}
\newcommand{\dif}{\backslash}
\newcommand{\Solution}{\noindent\textbf{Solution.}}
% \let\oldbigcup\bigcup
% \renewcommand{\bigcup}{\boldsymbol{\oldbigcup}} 

\newcommand{\ev}{\textsc{ev}}
\newcommand{\ii}{\hat{\imath}}
\newcommand{\jj}{\hat{\jmath}}
\newcommand{\kk}{\hat{k}}
\newcommand{\vv}{\mathbf{v}}
\newcommand{\va}{\mathbf{a}}
\newcommand{\vb}{\mathbf{b}}
\newcommand{\vd}{\mathbf{d}}
\newcommand{\vx}{\mathbf{x}}
\newcommand{\vy}{\mathbf{y}}
\newcommand{\vu}{\mathbf{u}}
\newcommand{\vzr}{\mathbf{0}}
\newcommand{\pp}{\boldsymbol{\cdot}}
\newcommand{\AND}{\quad\text{and}\quad}
\newcommand{\seq}{\subseteq}
\newcommand{\sep}{\supseteq}
\newcommand{\nseq}{\nsubseteq}
\newcommand{\then}{\implies}
\newcommand{\abs}[1]{\left|#1\right|}
\newcommand{\Abs}[1]{\left|\!\left|#1\right|\!\right|}
\newcommand{\prt}[1]{\left(#1\right)}
\newcommand{\End}{\hfill\(\square\)}
\newcommand{\Mod}[1]{\ \left(\mathrm{mod}\ #1\right)}
\newcommand{\floor}[1]{\left\lfloor #1 \right\rfloor}
\newcommand{\ceil}[1]{\left\lceil #1 \right\rceil}
\renewcommand\labelitemi{$\bullet$}
\newcommand{\interior}[1]{#1^\circ}
\renewcommand\labelitemi{$\bullet$}
\renewcommand{\labelitemii}{$ \circ $}
\renewcommand{\mathbb}[1]{\mathds{#1}}
\renewcommand*{\thefootnote}{\fnsymbol{footnote}}

\DeclareMathOperator{\lcm}{lcm}
\DeclareMathOperator{\adj}{adj}
\DeclareMathOperator{\sen}{sen}
\DeclareMathOperator{\proy}{proy}
\DeclareMathOperator{\tr}{tr}
\DeclareMathOperator{\spn}{span}
\DeclareMathOperator{\im}{Im}
\DeclareMathOperator{\dd}{d }
\DeclareMathOperator{\card}{card}
\DeclareMathOperator{\ddd}{d\!}
% \DeclareMathOperator{\y}{y}
\DeclareMathOperator{\rank}{rank}
\DeclareMathOperator{\nul}{nul}
\DeclareMathOperator{\Int}{int}
\DeclareMathOperator{\cla}{cla}

% Añadidos después
\newcommand{\vect}[1]{\left( {#1}_{1},\dots,{#1}_{n} \right)}

\usepackage{amsthm}
\newtheoremstyle{uptheorem} 
{0.75cm}{0.75cm}{\upshape}{}{\bfseries}{.}{ }{} 
\newtheoremstyle{sltheorem} 
{0.75cm}{0.3cm}{\slshape}{}{\bfseries}{.}{ }{} 
\theoremstyle{sltheorem}
\newtheorem{theorem}{Problem}%[chapter]
\newtheorem{definition}{Definition}%[chapter]
\newcommand{\definitionautorefname}{Definition}
\newtheorem{lemma}{Lemma}%[theorem]
\newcommand{\lemmaautorefname}{Lemma}
\newtheorem{corollary}{Corollary}%[theorem]
\newcommand{\corollaryautorefname}{Corollary}
\newtheorem{corolario}{Corolario}[theorem]
\newcommand{\corolarioautorefname}{Corolario}
\theoremstyle{uptheorem} 
\newtheorem{example}{Example}%[chapter]
  \newaliascnt{examples}{example}
  \newtheorem{examples}[examples]{Examples}
  \aliascntresetthe{examples}
  \def\exampleautorefname{Example}
  \def\examplesautorefname{Examples}

\newcommand{\notaautorefname}{Nota}

\newtheoremstyle{uptheoremv2}{0.7cm}{}{\slshape}{}{\bfseries}{.}{ }{} 
\theoremstyle{uptheoremv2} 
\newtheorem{problem}[theorem]{Problem}
\newcommand{\problemautorefname}{Problem}
\newtheoremstyle{uptheoremv3}{}{}{\upshape\small}{}{\bfseries}{.}{ }{} 
\theoremstyle{uptheoremv3}
% \newtheorem{remark}{Remark}[chapter]

\numberwithin{figure}{section}
\numberwithin{table}{section}
% \numberwithin{equation}{chapter}

% \usepackage[backend=biber]{biblatex}
% \bibliography{bib.bib}
% \nocite{*}

% 2024-01-26 Fri 00:23:25 -----------------------

\renewcommand{\thequestion}{\bfseries\arabic{question}}
\renewcommand{\solutiontitle}{\noindent\textit{Solution.}\enspace}
\unframedsolutions

% 2024-01-27 Sat 22:51:15 -----------------------
\footer{}{\thepage}{}

% 2024-02-02 Fri 00:46:15 --------------
% Proof in sol env
\newenvironment{theproof}
{
    \renewcommand{\solutiontitle}{}
    \begin{solution}
    \vspace*{-\baselineskip}
    \begin{proof}
}
{
    \end{proof}
    \end{solution}
    \renewcommand{\solutiontitle}{\noindent\textit{Solution.} }
}