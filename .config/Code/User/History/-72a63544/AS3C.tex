\chapter{Realization of a category}

We call \textit{geometric realization} to the operation that asigns a topological space to a simplicial complex. A more accurate name would be \textit{topological realization}

% the intuition 
% https://ncatlab.org/nlab/show/geometric+realization  What is commonly called geometric realization (or, less commonly but more accurately: topological realization,see Rem. 1.1) is the operation that builds from a simplicial set X a topological space |X|

\section{Simplicial sets}


In the following, we consider finite sets of the form \[\{0,1,\dots,n\}\quad\text{with}\quad n\in\N,\] 
When  endowed with the usual ordering of the natural numbers, these are totally ordered sets, and we denote them  by \([n]\). % for each \(n\in\N\).


\begin{definition}[Simplex category]
    The \textit{simplex category}\note{also called simplicial category, or nonempty finite ordinal category} \(\Delta\) consist of \begin{enumerate}[label=(\roman*)]
        \item Objects: the linearly ordered sets \([n]\), \(n\geq 0\).
        \item Morphisms: the order-preserving functions
    \end{enumerate}
\end{definition}

\begin{remark}
    The Morphisms are this way: \(f\) is a \(\Delta\)-morphism iff \(\dom f = [m]\)  and \(\cod f = [n]\) and \(m\leq n\) and \[
        0< 1 < \cdots < n \implies f(0)  f(1) < \cdots < f(n).
    \]
\end{remark}


\begin{definition}[Simplicial set]
    A simplicial set is a contravariant functor from \(\Delta\) to \(\mathbf{Set}\).
    
\end{definition}

\doubt{\url{https://math.stackexchange.com/questions/1528005/simplicial-complex-vs-delta-complex-vs-cw-complex}}

We can go a step further and define   \textit{simplicial objects}, which is done by replacing \(\mathbf{Set}\) with an arbitrary category in the above definition. 
For example, if we consider \(\mathbf{Grp}\), then we would talk about simplicial groups.

We are interested in simplicial sets because they helps to `model' topological spaces. 
[ritcher] Simplicial sets are particularly important because they model topological spaces

This is enough for our prposes.
but
a simplicial set is just a particular example of what is called a simplicial object, that is a contravariant functor from \(\Delta\) to some arbitrary category.

[Explain how we arrive to the classical definition from this one, but do not present it as an alternative definition, but as a way of specifying a simplicial set.]



\begin{remark}
    Given a simplicial set \(X\colon \Delta\to \mathbf{Set}\) and any \(\Delta\)-object \([n]\), 
    we denote \(X_n := X([n])\).
    Thus a simplicial set induces a sequence of sets \(\left( X_n \right)_{n\in\N}\).
\end{remark}

There are certain \(\mathbf{Set}\)-morphisms which are so important they get a particular name.

\begin{definition}[Coface maps]
    Let \(n\in \Z^+_0\). 
    For each \(i\in [n]\), 
    define a \(\Delta\)-morphism \(d^i\colon [n] \to [n+1]\) as follows:
    \[d^i(j) = \begin{cases}
        k& \text{if } j < i,\\
        k+1 & \text{if } j\geq  i.
    \end{cases}\]
    We call \(d^i\) the \(i\)th coface map.
\end{definition}

These are injective maps.
The \(i\)th coface map deletes the \(i\)th element in the image.

\begin{definition}[Codegeneracy maps]
    Let \(n\in \Z^+_0\). 
    For each \(i\in [n]\), 
    define a \(\Delta\)-morphism \(s^i\colon [n+1] \to [n]\) as follows:
    \[s^i(j) = \begin{cases}
        k& \text{if } j \leq i,\\
        k-1 & \text{if } j >  i.
    \end{cases}\]
    We call \(d^i\) the \(i\)th coface map.
\end{definition}

These are surjective maps.
The \(i\)th codegeneracy maps two elements to \(i\).

\begin{theorem}
    The coface and codegeneracy maps satisfy the folloing relationships.
    \begin{align*}
        d^id^j &= d^{j+1} d^i       \qquad\text{if } i\leq j,\\
        s^{j}s^i &= s^i s^{j+1}     \qquad\,\text{if } i\leq j,\\
        s^jd^i &= \begin{cases}
            d^i s^{j-1} &\text{ if } i <j, \\
            1 & \text{ if } i\in \{j,j+1\}, \\
            d^{i-1} s^j & \text{ otherwise }.
        \end{cases}
    \end{align*}
\end{theorem}
\begin{proof}
    
\end{proof}

\begin{lemma}
    Any \(\Delta\)-morphism has a unique factorization into composition of coface and codegeneracy maps, up to reordering.
\end{lemma}
\begin{proof}
    
\end{proof}

By the last lemma and last thm, a contravariant functor from \(\Delta\) is completely characterized by where it sends the objects  \([n]\)   and where it sends the coface and codegeneracy maps.

In fact, this give us a characterization of simplicial sets.

\begin{theorem}[Procedure to specify a simplicial set]
    Consider a sequence of sets  \((X_n)_{n\geq 0}\) and suppose that  
    \begin{enumerate}[label=(\roman*)]
        \item for each \(n\geq 0\) and  \(i\in [n]\), there is a map \(d_i\colon X_{n+1}\to X_n\), 
        \item for each \(n\geq 0\) and  \(i\in [n]\), there is a map \(s_i\colon X_{n}\to X_{n+1}\), and
        \item the following relations hold:
        \begin{align*}
            d^id^j &= d^{j+1} d^i       \qquad\text{if } i\leq j,\\
            s^{j}s^i &= s^i s^{j+1}     \qquad\,\text{if } i\leq j,\\
            s^jd^i &= \begin{cases}
                d^i s^{j-1} &\text{ if } i <j, \\
                1 & \text{ if } i\in \{j,j+1\}, \\
                d^{i-1} s^j & \text{ otherwise }.
            \end{cases}
        \end{align*}
    \end{enumerate}
    % Then there is a simplicial set. \(X\colon \Delta\to \mathbf{Set}\)
    This information gives rise to a simplicial set.
\end{theorem}

\section{The nerve of a category}

https://math.stackexchange.com/questions/2371516/geometric-realization-of-simplicial-sets

https://kerodon.net/tag/001X

see ritcher chap 11

% commented the folloing because it seems the coolor definition is using a functor. also maclane (p. 179) dice que the tradition way is more complicated, i.e. using the sequence is more complicated. 
% from curtis 71
% \begin{definition}[Simplicial set]
%     A simplicial set \(K\) is  a sequence of sets \(\left( K_n \right)_{n\in N}\) together with functions 
%     \begin{enumerate}[label=(\roman*)]
%         \item 
%     \end{enumerate}
% \end{definition}

% The category \(\mathbf{sSets}\) of simplicial sets.

The folloing is a topological definition.
\begin{definition}[Standard n-dimensional simplex]
    Let \(I = [0,1]\). The standard n-dimensional simplex is the set
    \[|{\Delta}^n| = \left\{ (x_0, \dots, x_n)\in I^{n+1 } :  x_0 + \cdots + x_n = 1 \right\}\]
    % This is a subspace endowed with the subspace topology.
    % Que representa..? el convex hull (de que?)
    % nos da un tetrahedro n dimensional ?
\end{definition}
\begin{remark}
    An alternative description of \(|\Delta^n|\)    can be given, namely as the convex hull of \(n+1\) points in \(\R^n\) with zeros every where but a single \(1\) in a component.
\end{remark}


Objects in \(\Delta\) have a geometric realization given by the  functor \(F\colon \Delta \to \mathbf{Top}\) such that 
\begin{itemize}
    \item \(F([n]) = |\Delta^n|\) for every \(\Delta\)-object \([n]\).
    \item \(F(f\colon [m] \to [n]) = f_*\colon |\Delta^m| \to |\Delta^n| \) for every \(\Delta\)-morphism \(f\colon [m]\to [n]\)
\end{itemize} 
Notice the \(i\)th vertex of \(|\Delta^n|\) is sent to the \(f(i)\)th vertex of \(|\Delta^m|\). 
Let's prove that \(F\) is indeed a functor.

\begin{remark}
    This construction must not be confused with the geometric realization of a simplicial set.
\end{remark}


\begin{definition}[Realization]
    Let \(X\) be a simplicial set, and endow \(X_n\) with the discrete topology.
    Let \(|\Delta^n|\) be the standard \(n\)-dimensional simplex in \(\R^{n+1}\).
    The \textit{geometric realization} of \(X\), denoted \(|X|\) is defined to be 
    \[|X| = \left( \coprod_{n\geq 0}  X_n \times  |\Delta^n| \right)/ \sim\]
    where \(|X|\) is endowed with the quotient topology, and \(\sim\) is an equivalence relation defined by \((x,p)\sim (y,q)\) if 
    \begin{enumerate}[label=(\roman*)]
        \item \(d_ix = y \) and \((d^iq = p)\), or 
        \item \(s_j x=y\) and \(s^j q = p\)
    \end{enumerate}
\end{definition}



% \section{The nerve of a category}


\begin{definition}
    Let \(\mathbf{C}\) be a category. ?The nerve of is a simplicial set s.t.

    category \(\mathcal{C}\),  let \(\mathrm{N}_n(\mathcal{C})\) denote the set of all functors from \([n]\) to \(\mathcal{C}\). view the construction \([n] \mapsto \mathrm{N}_n(\mathcal{C})\) as a simplicial set. We will denote this simplicial set by \(\mathrm{N}_{\bullet}(\mathcal{C})\) and refer to it as the nerve of \(\mathcal{C}\).
\end{definition}



Clasifying space of a category is the topological space











