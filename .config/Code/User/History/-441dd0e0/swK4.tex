\documentclass{article}
\usepackage{graphicx} % Required for inserting images
\usepackage{makeidx}
\usepackage{multirow} % Añadido para la función \multirow
\usepackage{amsmath, amssymb, multirow, float, hyperref, physics, subcaption, enumitem, csquotes}
\captionsetup[sub]{labelformat=brace}
\usepackage{amssymb}
\usepackage[spanish]{babel}
\selectlanguage{spanish}


\title{Construction of Simplicial Complexes (notes)}
\author
{John A. Moran }
\date{January 2024}



%----------------------------------------------------------------
\begin{document}
\maketitle
%----------------------------------------------------------------

\tableofcontents

% INtroducción
%%%%%%%%%%%%%%%%%%%%%%%%%%%%%%%%%%%%%%%%%%%%%%%%%%%%%%%%%%%%%%%%%%%%





\subsection{Complejos de Čech}

Muchos de los métodos topológicos usuales utilizan un complejo simplicial como entrada; sin embargo, no todos los objetos a estudiar vienen en esta forma. En esta sección se presentan diversos métodos para la construcción de complejos simpliciales asociados a una nube de puntos, es decir, vinculados a una colección finita de puntos en un espacio métrico. Estos puntos pueden representar una colección de datos numéricos como un subconjunto de $\mathbb{R}^n$.\\
\\
\textbf{Definición:} Sea $(X, d)$ un espacio métrico y sea $A \subset X$ una muestra finita de $X$. Considere el escalar $\epsilon \geq 0$. Se llama \textbf{Complejo de Čech} asociado a $A$ con parámetro $\epsilon$, denotado \textbf{Cech$(A, \epsilon)$}, al complejo simplicial definido por las siguientes reglas:
\begin{enumerate}
    \item El conjunto de vértices es $A$.
    \item Un subconjunto $\sigma = \{x_{0}, \ldots, x_{q}\} \subseteq A$ es un $q$-símplice si y solo si $\bigcap_{x \in \sigma} B(x, \epsilon) \neq \emptyset$.
\end{enumerate}
%%%%%%%%%%%%%%%%%%%%%%%%%%%%%%%%%%%%%%%%%%%%%%%%%%%%%%%%%%%%%%%%%%%%%%%
\begin{figure}[h]
  \centering
  \includegraphics[width=0.8\textwidth]{Complejo_de_Cech.png}
  \caption{Seis puntos en el plano y sus correspondientes 3 complejos de Čech Cech$(A, \epsilon)$. Nótese que para la aparición de un 2-símplice es necesaria la intersección de todas las bolas con centro en los correspondientes vértices.}
  \label{fig:mi_imagen}
\end{figure}
%%%%%%%%%%%%%%%%%%%%%%%%%%%%%%%%%%%%%%%%%%%%%%%%%%%%%%%%%%%%%%%%%%%%%%%
Nótese que $\bigcap_{x\in \sigma}B(x,\epsilon )\neq \emptyset$ implica la creación de un $n$-símplice cuando la intersección de $n$ bolas $B(x,\epsilon )$ es no vacía. Así, los radios de las bolas y sus intersecciones toman un papel importante en los complejos de Čech. Además, si $\sigma$ es un símplice, entonces lo es también cada uno de sus subconjuntos. Por tanto, los complejos de Čech son también complejos simpliciales abstractos. Debido a su buena interpretación geométrica y su utilidad en varios contextos topológicos, los complejos de Čech resultan interesantes, incluso cuando su computación resulta ser todo un desafío.\\
\\
\textbf{Definición:} Una \textbf{filtración} de un complejo simplicial $K$ es una secuencia de subcomplejos encajados de $K$: 
\begin{align*}
    \emptyset = K^{0}\subset K^{1}\subset ... \subset K^{n}=K
\end{align*}
Una filtración puede ser vista como una construcción en la cual nuevos símplices van siendo agregados en cada etapa, donde, para que un símplice $\sigma$ se forme, primero deben formarse todas sus caras. Nótese que la filtración empieza con el complejo vacío $K^{0}=\emptyset$ y termina con el complejo completo $K$.
%%%%%%%%%%%%%%%%%%%%%%%%%%%%%%%%%%%%%%%%%%%%%%%%%%%%%%%%%%%%%%%%%%%%%%%
\begin{figure}[h]
  \centering
  \includegraphics[width=0.8\textwidth]{Filtración.png}
  \caption{Filtración de un complejo simplicial cualquiera.}
  \label{fig:mi_imagen}
\end{figure}
%%%%%%%%%%%%%%%%%%%%%%%%%%%%%%%%%%%%%%%%%%%%%%%%%%%%%%%%%%%%%%%%%%%%%%%
\textbf{Definición:} Sea $(X, d)$ un espacio métrico y sea $A \subset X$ una muestra finita de $X$. La \textbf{filtración de Čech} en $A$ es la colección de los complejos simpliciales abstractos $\{Cech(A,\epsilon)\}_{\epsilon \geq 0}$ donde
\begin{align*}
    Cech(A,\epsilon_{i}) \subseteq Cech(A,\epsilon_{j}), \text{ para todo } i < j.
\end{align*}
Una filtración de Čech proporciona la colección de todos los complejos de Čech en $A$, mientras que un único complejo de Čech depende de la elección de la escala $\epsilon$.





\subsection{Complejos de Rips}




A pesar de sus buenas interpretaciones geométricas y propiedades topológicas, en la práctica, el complejo de Čech resulta ser computacionalmente inmanejable. Por ejemplo, es posible que muchas bolas $B(x,\epsilon)$ se superpongan, generando así símplices innecesarios de diferentes dimensiones que solo consumen espacio de almacenamiento.\\
\\
Una solución a este problema es reconstruir el complejo utilizando únicamente información sobre la distancia entre sus vértices. De esta manera, no es necesario verificar las intersecciones no vacías en todas las subcolecciones de $B(x,\epsilon)$. Una variante del complejo de Čech es el complejo de Rips, que implementa esta solución.\\
\\
\textbf{Definición:} Sea $(X, d)$ un espacio métrico y sea $A \subset X$ una muestra finita de $X$. Considere el escalar $\epsilon \geq 0$. Se llama \textbf{Complejo de Rips} asociado a $A$ con parámetro $\epsilon$, denotado $Rips(A,\epsilon)$, al complejo simplicial definido por las siguientes reglas:
\begin{enumerate}
    \item El conjunto de vértices es $A$.
    \item Un subconjunto $\sigma = \{x_{0}, \ldots , x_{q}\} \subseteq A$ es un $q$-símplice si y solo si $Diam(\sigma)\geq \epsilon$.
\end{enumerate}
%%%%%%%%%%%%%%%%%%%%%%%%%%%%%%%%%%%%%%%%%%%%%%%%%%%%%%%%%%%%%%%%%%%%%%%
\begin{figure}[h]
  \centering
  \includegraphics[width=0.8\textwidth]{Complejo_de_Rips.png}
  \caption{Seis puntos en el plano y sus correspondientes 3 complejos de Rips \textbf{Rips$(A, \epsilon)$}. Nótese que para la aparición de un $n$-símplice se necesita que la distancia entre cualquier par de sus $n+1$ vértices sea a lo sumo $2\epsilon$.}
  \label{fig:mi_imagen}
\end{figure}
%%%%%%%%%%%%%%%%%%%%%%%%%%%%%%%%%%%%%%%%%%%%%%%%%%%%%%%%%%%%%%%%%%%%%%%
En algunas ocasiones, los complejos de Rips también se llaman complejos de Vietoris-Rips. Nótese que $Diam(\sigma)\geq \epsilon$ implica que la distancia entre cualquier par de vértices de $\sigma$ es a lo sumo $\epsilon$. Además, si $\sigma$ es un símplice, entonces cada uno de sus subconjuntos también lo es. En otras palabras, los complejos de Rips son, de hecho, complejos simpliciales abstractos.\\
\\
\textbf{Definición:} Sea $(X, d)$ un espacio métrico y sea $A \subset X$ una muestra finita de $X$. La \textbf{filtración de Rips} en $A$ es la colección de los complejos simplicales abstractos $\{Rips(A,\epsilon)\}_{\epsilon \geq 0}$ donde
\begin{align*}
    Rips(A,\epsilon_{i}) \subseteq Rips(A,\epsilon_{j}), \text{ para todo } i < j
\end{align*}
Aunque los conjuntos de vértices de los complejos de Čech y Rips son idénticos, la diferencia radica en los distintos valores que se le asignen a $\epsilon$, ya que de este valor depende la creación de nuevas caras, y por ende, la creación de nuevos complejos simpliciales. Como resultado, $Rips(A,\epsilon) \subseteq Cech(A,\epsilon)$. O sea, el complejo de
Vietoris-Rips tiene menos o igual número de símplices que el complejo de Čech para un mismo parámetro.\\
\\
(gráfico comparacion entre los complejos)\\
\\


\subsection{Nervio de $\mathcal{U}$}

El concepto del nervio resulta útil en la construcción de complejos simpliciales ya que este es un tipo especial de recubrimiento.\\
\\
\textbf{Definición:} Sea $X$ un espacio topológico y $\mathcal{U}=\{U_{i}\}_{i \in I}$ un recubrimiento abierto de $X$. Se dice que $\mathcal{U}$ es un \textbf{buen recubrimiento de $X$} si se cumple que
\begin{enumerate}
    \item Todos los abiertos $U_{i}$ pertenecen a $\mathcal{U}$.
    \item La intersección finita y no vacía de q-elementos de $\mathcal{U}$, $U_{i_{1}}\cap \ldots \cap U_{i_{q}}$, es contractil.
\end{enumerate}
\textbf{Definición:} Sea $X$ un espacio topológico y sea $\mathcal{U}=\{U_{i}\}_{i \in I}$ un recubrimiento de $X$. El \textbf{nervio} de $\mathcal{U}$, denotado \textbf{$\mathcal{N}(\mathcal{U})$}, es el complejo simplicial abstracto definido por las siguientes reglas:
\begin{enumerate}
    \item El conjunto de vértices es $\mathcal{U}=\{U_{i}\}_{i \in I}$.
    \item Un subconjunto $\sigma = \{U_{i_{1}}, U_{i_{2}}, ... ,U_{i_{q}}\} \subseteq \mathcal{U}$ es un q-símplice si y solo si $U_{i_{1}}\cap U_{i_{2}} \ldots \cap U_{i_{q}} \neq \emptyset$.
\end{enumerate}
%%%%%%%%%%%%%%%%%%%%%%%%%%%%%%%%%%%%%%%%%%%%%%%%%%%%%%%%%%%%%%%%%%%%%%%
\begin{figure}[h]
  \centering
  \includegraphics[width=0.8\textwidth]{Nervio_U.png}
  \caption{Ejemplo de un conjunto $\mathcal{U}$ y su respectivo nervio.}
  \label{fig:mi_imagen}
\end{figure}
%%%%%%%%%%%%%%%%%%%%%%%%%%%%%%%%%%%%%%%%%%%%%%%%%%%%%%%%%%%%%%%%%%%%%%%
Se entiende que los conjuntos de $\mathcal{U}$ no son vacíos. Antes de continuar es necesario recurrir al siguiente lema:\\
\\
\textbf{Lema:} Todo subconjunto convexo de $\mathbf{R}^n$ es contractil.\\
\\
Una demostración de este lema se puede encontrar en [Citar la dem.]. Ahora, dado que la colección $\mathcal{U} = {B(a,\epsilon)}_{a \in A}$ es un buen recubrimiento de $A$, ya que toda bola en $\mathbf{R}^n$ es un conjunto convexo, entonces el complejo de Čech es el nervio correspondiente a dicha colección, i.e., 
\begin{align*}
    Cech(A,\epsilon) = \mathcal{N}(\{B(a,\epsilon)\}_{a \in A})
\end{align*}


\textbf{Teorema:} \textbf{[Teorema del nervio]} Sea $\mathcal{U}=\{U_{1}, U_{2}, ... ,U_{k}\}$ un recubrimiento finito contable de subconjuntos convexos cerrados de $\mathbf{R}^n$. Entonces, 
\begin{align*}
    \bigcup_{i=1}^{k}U_{k} \simeq \mathcal{N}(\mathcal{U})
\end{align*}
La demostración de este teorema se puede encontrar en [Citar la dem.]. 



\section{Complejo de Delaunay}
Pese a su eficiencia sobre el complejo de Čech, el complejo de Rips sigue siendo computacionalmente complicado. A medida que la nube de puntos aumenta, aparecen aglomeraciones donde no interesa conocer en detalle qué pasa en cada uno de sus puntos. Por tanto, se propone como solución \texttt{
``agrupar''} aglomeraciones de puntos cercanos en uno solo. De esta manera, se reduce la carga computacional sin afectar la estructura topológica. Por ejemplo, para representar la esfera $\mathcal{S}^1$ se necesita una infinidad de 0-símplices. Nótese que como ventaja adicional se logra de cierta manera discretizar espacios continuos. De aquí en adelante, $\Gamma$ denotará un conjunto de puntos de referencia.\\
\\
\textbf{Definición:} Sea $\Gamma \subset X$ un subconjunto no-vacío de $X$. Dado un punto de referencia $\gamma \in \Gamma$. Se define la \textbf{Celda de Voronoi} asociadada a $\gamma$ como
\begin{align*}
    V_{\gamma}=\{x \in X: \text{ } d(x,\gamma) \leq d(x,\lambda) \text{ para todo } \lambda \in X\backslash A \{ \lambda\}\}
\end{align*}
Nótese que las celdas de Voronoi descomponen a $X$ en disitintas regiones, formando además un recubrimiento no abierto de $X$. Para la creación del siguiente complejo simplicial, las aglomeraciones de puntos quedan \texttt{``encapsuladas''} en una celda de Voronoi, de esta manera, todos los puntos de la celda se sustituyen por el de referencia. Cabe recalcar que la intersección entre dos celdas es no-vacía, ya que cada celda incluye su borde.\\
\\
\textbf{Definición:} Se dice \textbf{complejo de Delaunay} asociado a $\Gamma$ al nervio del recubrimiento dado por las celdas de Voronoi.%%%%%%%%%%%%%%%%%%%%%%%%%%%%%%%%%%%%%%%%%%%%%%%%%%%%%%%%%%%%%%%%%%%%%%%
\begin{figure}[h]
  \centering
  \includegraphics[width=0.8\textwidth]{Complejo_de_Voronoi.png}
  \caption{Izquierda: Nube de puntos en el plano con cuatro agrupaciones representadas por sus puntos de referencia marcados con diferentes colores. Centro: Celdas de Voronoi para los cuatro puntos de referencia. Derecha: Complejo de Delaunay correspondiente a las celdas creadas.}
  \label{fig:mi_imagen}
\end{figure}
%%%%%%%%%%%%%%%%%%%%%%%%%%%%%%%%%%%%%%%%%%%%%%%%%%%%%%%%%%%%%%%%%%%%%%%
\section{Complejo de Vietoris}
\textbf{Definición:} Sea $\mathcal{U}=\{U_{1}, U_{2}, ... ,U_{k}\}$ una colección contable de subconjuntos de un espacio finito $X$ tal que $U_{1}\cup ... \cup U_{k}=X$. El \textbf{complejo de Vietoris} de $\mathcal{U}$, denotado \textbf{$\mathcal{V}(\mathcal{U})$}, es el complejo simplicial abstracto definido por las siguientes reglas:
\begin{enumerate}
    \item El conjunto de vértices es $\mathcal{U}$.
    \item Un subconjunto $\sigma \subseteq X$ es un símplice si y solo si existe un $U \in \mathcal{U}$ tal que $\sigma \subseteq U$.
\end{enumerate}
%%%%%%%%%%%%%%%%%%%%%%%%%%%%%%%%%%%%%%%%%%%%%%%%%%%%%%%%%%%%%%%%%%%%%%%
\begin{figure}[h]
  \centering
  \includegraphics[width=0.8\textwidth]{Complejo_Vietoris.png}
  \caption{Recubrimiento $\mathcal{U}$ de un conjunto de ocho puntos y su respectivo complejo de Vietoris $\mathcal{V}(\mathcal{U})$.}
  \label{fig:mi_imagen}
\end{figure}
%%%%%%%%%%%%%%%%%%%%%%%%%%%%%%%%%%%%%%%%%%%%%%%%%%%%%%%%%%%%%%%%%%%%%%%
Mientras que los complejos de Čech son nervios associados a la colección $\{Rips(A,\epsilon)\}_{\epsilon \geq 0}$, los complejos de Rips son Complejos de Vietoris asociados a la colección $\{U\}_{Diam(U)\leq \epsilon}$.

\end{document}
