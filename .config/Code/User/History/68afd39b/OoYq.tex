\usepackage[T1]{fontenc}
\usepackage{fontspec}
% \usepackage[utf8]{inputenc}
\usepackage[english]{babel}

\usepackage{
    geometry,
    graphicx,
    lmodern,
    microtype,
    csquotes,
    lipsum,
    anyfontsize,
    url,
    titlesec,
    fancyhdr,
    calc,
    ragged2e,
    caption,
    amsmath,
    amsfonts,
    mathtools,
    amsthm,
    amssymb,
    xcolor,
    thmtools,
    mdframed,
    tcolorbox,
    enumitem,
    dsfont,
}
\tcbuselibrary{theorems}

% --------------------------------------------
% Theorem-like environments

% \newtheorem{theorem}{Theorem}[chapter] 
% \newtheorem{definition}[theorem]{Definition}
% \newtheorem{proposition}[theorem]{Proposition} 
% \newtheorem{lemma}[theorem]{Lemma} 
% \newtheorem{example}[theorem]{Example} 


%% This is from the experiments folder

\tcbset{
    mystyle/.style={
        colback=white, % Transparent background
        colframe=black, % Frame color
        coltitle=black, % Title color (adjust as needed)
        colbacktitle=white,
        fonttitle=\sectionfontc,
        sharp corners, % Sharp corners
        boxrule=0.5pt, % Frame thickness
        left=3pt, % Reduce left inner space
        right=3pt, % Reduce right inner space
        top=3pt, % Reduce top inner space
        bottom=3pt, % Reduce bottom inner space
        toptitle=3pt,bottomtitle=2pt,
        separator sign=\hspace*{10pt},
        before skip=\baselineskip, % Increase space before the box
        after skip=\baselineskip, % Increase space after the box
        description font=\sectionfontc\itshape,   
        % number format=\thesection.\arabic{mytheorem},
    }
}


\newtcbtheorem[number within=chapter]{theorem}{Theorem}{mystyle}{theorem}
\newtcbtheorem[use counter from=theorem]{definition}{Definition}{mystyle}{definition}
\newtcbtheorem[use counter from=theorem]{proposition}{Proposition}{mystyle}{proposition}
\newtheoremstyle{uptheorem} 
{\baselineskip}{\baselineskip}{\upshape}{}{\bfseries}{.}{ }{} 
\theoremstyle{uptheorem}
\newtheorem{example}[tcb@cnt@theorem]{Example}
\newtheorem{lemma}[tcb@cnt@theorem]{Lemma}

\newtheoremstyle{uptheoremv3}{\baselineskip}{}{\upshape\small}{}{\bfseries}{.}{ }{} 
\theoremstyle{uptheoremv3}
\newtheorem{remark}[tcb@cnt@theorem]{Remark}





% --------------------------------------------
% Custom font

\newfontfamily\sectionfontc{NimbusSanL.ttf}[
  Path = nimbussanl/,
  Extension = .ttf,
  UprightFont = *,  
  BoldFont = *-Bold,
  ItalicFont = *-Italic,
  BoldItalicFont = *-BoldItalic,
  Scale = 0.92,
]

\titleformat{\chapter}[display]
  {\sectionfontc\mdseries\Huge}
  {\huge\chaptertitlename\ \thechapter}
  {1em}
  {}

\titleformat{\section}
  {\sectionfontc\mdseries\huge}
  {\thesection}
  {1em}
  {}

\titleformat{\subsection}
  {\sectionfontc\mdseries\Large}
  {\thesubsection}
  {1em}
  {}

\titleformat{\subsubsection}
  {\sectionfontc\mdseries}
  {\thesubsubsection}
  {1em}
  {}

\captionsetup{
    labelfont=bf,
    font=footnotesize,
    justification=raggedright, % Left-align the caption
    singlelinecheck=false, % Disable single line check for multi-line captions
}
% --------------------------------------------
% Layout & measures

\raggedbottom
\setlength{\parskip}{0.5\baselineskip}
\footmarkstyle{\textsuperscript{#1}\;}    % indent the footmark 
\setlength{\footmarkwidth} {0em}
\setlength{\footmarksep}   {0em}



\geometry{
    % showframe,
    margin=1.5cm,
    includeheadfoot,
    includemp,
    marginparsep=0.75cm,
    marginparwidth=5.25cm
}

\pagestyle{fancy}
\fancyhead{} % clear all header fields
\fancyhead[LE]{\sectionfontc\mdseries\nouppercase{\leftmark}}
\fancyhead[RO]{\sectionfontc\mdseries\nouppercase{\rightmark}}
\fancyfoot{} % clear all footer fields
\fancyfoot[LE,RO]{\thepage}
\renewcommand{\headrulewidth}{0pt}

\fancypagestyle{plain}{% % <-- this is new
  \fancyhf{} 
  \fancyfoot[LE,RO]{\thepage} % same placement as with page style "fancy"
  \renewcommand{\headrulewidth}{0pt}}

\footnotesinmargin

% SIDECAPTIONS
\setsidecaps{\marginparsep}{\marginparwidth}
\sidecapmargin{outer}
\setsidecappos{t}
\renewcommand*{\sidecapstyle}{%
\captionnamefont{\bfseries\foottextfont}
\captionstyle{\RaggedRight\footnotesize\foottextfont}
}

% FULLWIDTH environment
% The following code should be used *after* any changes to the margins and
% page layout are made (e.g., after the geometry package has been loaded).
\newlength{\fullwidthlen}
\setlength{\fullwidthlen}{\marginparwidth}
\addtolength{\fullwidthlen}{\marginparsep}

\newenvironment{fullwidth}{%
  \begin{adjustwidth*}{}{-\fullwidthlen}%
}{%
  \end{adjustwidth*}%
}

% Custom commands --------------------------------------------

\newcommand{\qand}{\quad\text{and}\quad}
\newcommand{\chr}[1]{\textcolor{red}{#1}}
\newcommand{\R}{\mathbb{R}}
\newcommand{\Q}{\mathbb{Q}}
\renewcommand{\mathbb}[1]{\mathds{#1}}
\renewcommand{\subset}{\subseteq}



% --------------------------------------------

\usepackage{hyperref}
\hypersetup{
    bookmarksnumbered,
    hidelinks,
    colorlinks=true,
    allcolors=blue,
}

% --------------------------------------------
% Bibliography

\usepackage[backend=biber,style=numeric]{biblatex}
\addbibresource{bibliography.bib}


