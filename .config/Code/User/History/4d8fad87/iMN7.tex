% \DeclareMathSizes{12}{14.5}{9}{9}
\usetheme[]{Warsaw}
\usecolortheme{whale}
\useoutertheme{infolines}
% \beamertemplateshadingbackground{white!90}{blue!05}
\setbeamersize{text margin left=6mm,text margin right=6mm} 
\beamertemplatenavigationsymbolsempty
\setbeamertemplate{caption}[numbered]
\setbeamercovered{transparent}
\setbeamertemplate{theorems}[numbered]
\usebackgroundtemplate{
    \tikz\node[opacity=0.15]{
    \hspace*{-3mm} 
    \includegraphics[height=1.1\textheight]{images/background_white.pdf}
};}
% \usefonttheme{professionalfonts}

\usepackage[T1]{fontenc}
\usepackage[utf8]{inputenc}
\usepackage{
    amsmath,
    amsthm,
    amsfonts,
    graphicx,
    polynom,
    lipsum,
    % lmodern,
    systeme,
    float,
    ifthen,
    mathrsfs,
    units,
    enumerate,
    % enumitem,
    multicol,
    geometry,
    mdframed,
    setspace,
    caption,
    anyfontsize,
    mathtools,
    fancyhdr,
    booktabs,
    multirow,
    csquotes,
    microtype,
    bookmark,
    emptypage,
    blkarray,
    aliascnt,
    dsfont,
    comment,
    tabularx,
    % mathds,
    tikz,
    listings,
    textcomp,
    % cmbright,
}
\usepackage[scaled=0.95]{helvet}
\renewcommand{\mathbb}[1]{\mathds{#1}}
\newcommand{\N}{\mathbb{N}}
\newcommand{\Z}{\mathbb{Z}}
\newcommand{\R}{\mathbb{R}}
\newcommand{\Q}{\mathbb{Q}}
\newcommand{\II}{\mathbb{I}}
\newcommand{\K}{\mathbb{K}}
\newcommand{\PP}{\mathbb{P}}
\newcommand{\C}{\mathbb{C}}
\newcommand{\T}{\mathscr{T}}
\newcommand{\I}{\mathcal{I}}
\newcommand{\E}{\mathscr{E}}
\newcommand{\B}{\mathscr{B}}
\newcommand{\Fam}{\mathscr{F}}
\newcommand{\Nei}{\mathscr{N}}
\newcommand{\Pt}{\mathscr{P}}
\newcommand{\ES}{\text{\upshape\O}}
\newcommand{\TopS}{\left( X, \T \right)}
\newcommand{\bcap}{\bigcap}
\newcommand{\bcup}{\bigcup}
\newcommand{\dif}{\backslash}
\renewcommand{\ES}{\varnothing}

\newcommand{\FIP}{\textsc{fip}}
\newcommand{\ii}{\hat{\imath}}
\newcommand{\jj}{\hat{\jmath}}
\newcommand{\kk}{\hat{k}}
\newcommand{\vv}{\mathbf{v}}
\newcommand{\va}{\mathbf{a}}
\newcommand{\vb}{\mathbf{b}}
\newcommand{\vd}{\mathbf{d}}
\newcommand{\vx}{\mathbf{x}}
\newcommand{\vy}{\mathbf{y}}
\newcommand{\vu}{\mathbf{u}}
\newcommand{\vzr}{\mathbf{0}}
\newcommand{\pp}{\boldsymbol{\cdot}}
\newcommand{\AND}{\quad\text{and}\quad}
\newcommand{\seq}{\subseteq}
\newcommand{\sep}{\supseteq}
\newcommand{\nseq}{\nsubseteq}
\newcommand{\then}{\implies}
\newcommand{\abs}[1]{\left|#1\right|}
\newcommand{\Abs}[1]{\left|\!\left|#1\right|\!\right|}
\newcommand{\prt}[1]{\left(#1\right)}
\newcommand{\End}{\hfill\(\square\)}
\newcommand{\Mod}[1]{\ \left(\mathrm{mod}\ #1\right)}
\newcommand{\floor}[1]{\left\lfloor #1 \right\rfloor}
\newcommand{\ceil}[1]{\left\lceil #1 \right\rceil}
% \renewcommand\labelitemi{$\bullet$}
% \renewcommand{\labelitemii}{$ \circ $}
\renewcommand{\mathbb}[1]{\mathds{#1}}
\renewcommand*{\thefootnote}{\fnsymbol{footnote}}
 
 
\usepackage{amsthm}
\newtheoremstyle{uptheoremv3}{}{}{\upshape\small}{}{\bfseries}{.}{ }{} 
\theoremstyle{uptheoremv3}
\newtheorem{remark}{Remark}%[chapter]
% \newtheoremstyle{uptheorem} 
% {0.75cm}{0.75cm}{\upshape}{}{\bfseries}{.}{ }{} 
% \newtheoremstyle{sltheorem} 
% {0.75cm}{0.3cm}{\slshape}{}{\bfseries}{.}{ }{} 
% \theoremstyle{sltheorem}
% \newtheorem{theorem}{Theorem}%[chapter]
% \newtheorem{lemma}{Lemma}
% \newcommand{\lemmaautorefname}{Lemma}
% % \newtheorem{definition}{Definition}%[chapter]
% \newcommand{\definitionautorefname}{Definition}
% % \newtheorem{corollary}{Corollary}%[theorem]
% \newcommand{\corollaryautorefname}{Corollary}

\usepackage{hyperref}
% \hypersetup{%
%     pdftitle    ={},
%     pdfauthor   ={Christian Chávez},
%     pdfsubject  ={Topology},
%     pdfpagemode=UseNone,
% 		% colorlinks,
% 		% linkcolor = {blue},
% 		% allcolors = {blue},
% }

\usepackage{cleveref} 
\let\chyperref\cref % Save the orginal command under a new name
\renewcommand{\cref}[1]{\hyperlink{#1}{\chyperref{#1}}} % Redefine the \cref command and explictely add the hyperlink. 

\crefname{theorem}{Lemma}{Lemmas}
% \Crefname{lemma}{Lemma}{Lemmas}
\crefname{enumi}{item}{items}
\Crefname{enumi}{item}{items}


\usepackage[delims=\lbrack\rbrack]{spalign}
\let\matrix=\spalignmat
\let\Amatrix=\spalignaugmat
\let\AAmatrix=\spalignaugmatn
\newcommand{\Dmatrix}[1]{\spaligndelims\vert\vert\spalignmat{#1}}



\usepackage[noend]{algpseudocode}
\usepackage{listings}
\usepackage{xcolor}
\captionsetup[figure]{font=footnotesize,labelfont=bf}
\definecolor{codegreen}{rgb}{0,0.6,0}
\definecolor{codegray}{rgb}{0.5,0.5,0.5}
\definecolor{codepurple}{rgb}{0.58,0,0.82}
\definecolor{backcolour}{rgb}{0.95,0.95,0.92}
\definecolor{myblue}{RGB}{33, 111, 255}
\definecolor{myorange}{RGB}{255, 177, 33}
\usepackage{accsupp}
\newcommand{\noncopynumber}[1]{%
    \BeginAccSupp{method=escape,ActualText={}}%
    #1%
    \EndAccSupp{}%
}
\lstdefinestyle{mystyle}{
    backgroundcolor=\color{backcolour},   
    commentstyle=\color{codegreen},
    keywordstyle=\color{blue},
    stringstyle=\color{orange},
    basicstyle=\ttfamily\small,
    basewidth=0.49em,
    % otherkeywords={==,<,>,~},
    % keywordstyle=\color{orange},
    breakatwhitespace=false,         
    breaklines=true,                 
    captionpos=b,                    
    keepspaces=true,                 
    % numbers=left,                    
    numbersep=10pt,                  
    showspaces=false,                
    showstringspaces=false,
    showtabs=false,                  
    tabsize=1,
    mathescape =true,
    inputencoding=utf8,
    extendedchars=true,
    numbers=left,
    xleftmargin=3em,
    frame=single,
    framexleftmargin=2.5em,
    numberstyle=\small\color{codegray}\noncopynumber,
}

\lstset{style=mystyle,upquote=true,language=Matlab}
\newcommand{\tria}{\text{\textsl{Tria}}}
\newcommand{\adj}{\text{\textsl{Adj}}}
\newcommand{\ptt}{\text{\textsl{Point}}}
\DeclareMathOperator{\dd}{d\!}


\usepackage[backend=biber,style=bwl-FU]{biblatex}
\addbibresource{references.bib}
\setlength\bibitemsep{0.75\baselineskip}

%%%%%%%%%%%%%%%%%%%%%%%%%%%%%%%%%%%%%%%%%%%%%%
%%%%%%%%%%%          START        %%%%%%%%%%%%
%%%%%%%%%%%%%%%%%%%%%%%%%%%%%%%%%%%%%%%%%&&&&&
