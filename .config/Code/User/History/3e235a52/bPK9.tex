\documentclass[12pt]{article}

\usepackage{mathptmx}%
\usepackage{tabularx}
\usepackage{ltablex}
\usepackage{color}


    \begin{document}

    This document is a list of changes made according to the requests for both examiners. Texts in black are the requested items from examiners, while texts in \textcolor{blue}{blue} are the amened items. Pages refer to the page number on which these requested items are in both the original document and \textcolor{blue}{amended document}.\\
    \newline
    \newline
    \begin{center}
    \begin{tabularx}{\linewidth}{ c | X }


    \multicolumn{2}{c}{\textbf{Changes Required from Examiner A}} \\
    \hline
    \hline
    \textbf{\textit{page}} & \textbf{\textit{Content}} \\
    \hline

       i & A good English speaker should look at the title, the Abstract, and Introduction and eliminate any
    elementary mistakes of no/definite/indefinite article, subject/verb agreement.\\
    & \\
       \textcolor{blue}{i} & \textcolor{blue}{All grammar has been checked.} \\
       & \\
    \hline

    N/A & The candidate should consider all titles of sections that are uninformative to any other reader, such as 4.2 Test WaveFlume, 4.3 Test Fiel\_H5\_T10, 4.4 Test Field\_IHFOAM, 4.5 Test Field\_Extend, 4.6
    Test Field\_H2\_T20, 5.1 Test\_SHORELINE. They are all meaningless, except to the person who has
    done the work. I strongly suggest something like “Flat bed, single water layer", "Flat bed, water
    layer and air layer", and so on.   \\
    & \\
     \textcolor{blue}{N/A} & \textcolor{blue}{All titles have been changed accordingly as follows:
    3.1 Laboratory Flat Bed Wave Flume Test , 
    3.2 Field Scale Flat Bed Test ,
    3.3 Field Scale Flat Bed Test using IHFOAM ,
    3.4 Field Scale Flat Bed Test using FoamExtend ,
    3.5 Field Scale Flat Bed Test using Different Wave Parameters , 
    4 Continental Shelf Scale Simulation Series ,
    4.1 Continental Shelf Scale Test with A Constant Slope , 
    4.2 Continental Shelf Scale Test Validation  ,
    4.2.1
    Continental Shelf Scale Test with A Varying Cell Thickness in Wave Propagating
    Direction ,
    4.2.2 Flat Bed Continental Shelf Scale Test ,
    4.2.3 Continental Shelf Scale Test with Fluid Viscosity.} \\
    & \\   
    \hline    
      N/A & The candidate should insert dimensionless wave parameters wherever a wave is described, e.g. 1
    m, 10 s waves. I would prefer using wavelength, such as using values of $H/L$ and $L/d$, but the
    candidate might prefer to use $gT^2$ as a length scale. I always find it not very informative, and it is
    ambiguous anyway, as $T$ depends on wave speed.  \\
    & \\
    \textcolor{blue}{N/A} & \textcolor{blue}{Dimensionless terms have been included throughout the report, with an additional table, Table 3.3, summarises all wave parameters and relevant dimensionless terms used in the report.}\\ 
    & \\
    \hline
     38 & Referring to the examiner’s comment above about the outdated Figure 4.9, one could say that
    including it was redundant and the candidate might have shown more critical sense.
    The candidate could remove Figure 4.9 from the thesis. If that is not done, at least he should correct
    cnodial to cnoidal in two places in the caption.\\
     & \\
    \textcolor{blue}{37 - 39} & \textcolor{blue}{Figure 4.9 is removed, while additional information is added to make to conclusion with more critical thoughts.}\\    
       i & A good English speaker should look at the title, the Abstract, and Introduction and eliminate any
    elementary mistakes of no/definite/indefinite article, subject/verb agreement.\\
    & \\
       \textcolor{blue}{i} & \textcolor{blue}{All grammar has been checked.} \\
       & \\
    \hline

    N/A & The candidate should consider all titles of sections that are uninformative to any other reader, such as 4.2 Test WaveFlume, 4.3 Test Fiel\_H5\_T10, 4.4 Test Field\_IHFOAM, 4.5 Test Field\_Extend, 4.6
    Test Field\_H2\_T20, 5.1 Test\_SHORELINE. They are all meaningless, except to the person who has
    done the work. I strongly suggest something like “Flat bed, single water layer", "Flat bed, water
    layer and air layer", and so on.   \\
    & \\
     \textcolor{blue}{N/A} & \textcolor{blue}{All titles have been changed accordingly as follows:
    3.1 Laboratory Flat Bed Wave Flume Test , 
    3.2 Field Scale Flat Bed Test ,
    3.3 Field Scale Flat Bed Test using IHFOAM ,
    3.4 Field Scale Flat Bed Test using FoamExtend ,
    3.5 Field Scale Flat Bed Test using Different Wave Parameters , 
    4 Continental Shelf Scale Simulation Series ,
    4.1 Continental Shelf Scale Test with A Constant Slope , 
    4.2 Continental Shelf Scale Test Validation  ,
    4.2.1
    Continental Shelf Scale Test with A Varying Cell Thickness in Wave Propagating
    Direction ,
    4.2.2 Flat Bed Continental Shelf Scale Test ,
    4.2.3 Continental Shelf Scale Test with Fluid Viscosity.} \\
    & \\   
    \hline    
      N/A & The candidate should insert dimensionless wave parameters wherever a wave is described, e.g. 1
    m, 10 s waves. I would prefer using wavelength, such as using values of $H/L$ and $L/d$, but the
    candidate might prefer to use $gT^2$ as a length scale. I always find it not very informative, and it is
    ambiguous anyway, as $T$ depends on wave speed.  \\
    & \\
    \textcolor{blue}{N/A} & \textcolor{blue}{Dimensionless terms have been included throughout the report, with an additional table, Table 3.3, summarises all wave parameters and relevant dimensionless terms used in the report.}\\ 
    & \\
    \hline
     38 & Referring to the examiner’s comment above about the outdated Figure 4.9, one could say that
    including it was redundant and the candidate might have shown more critical sense.
    The candidate could remove Figure 4.9 from the thesis. If that is not done, at least he should correct
    cnodial to cnoidal in two places in the caption.\\
     & \\
    \textcolor{blue}{37 - 39} & \textcolor{blue}{Figure 4.9 is removed, while additional information is added to make to conclusion with more critical thoughts.}\\    

    \end{tabularx}

    \end{center}
    \end{document}