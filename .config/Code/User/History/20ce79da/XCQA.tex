\usepackage[T1]{fontenc}
\usepackage{fontspec}
% \usepackage[utf8]{inputenc}
\usepackage[english]{babel}

\usepackage{
    geometry,
    graphicx,
    lmodern,
    microtype,
    csquotes,
    lipsum,
    anyfontsize,
    url,
    titlesec,
    fancyhdr,
    calc,
    ragged2e,
    caption,
    amsmath,
    amsfonts,
    mathtools,
    amsthm,
    amssymb,
    xcolor,
    thmtools,
    % mdframed,
    % tcolorbox,
    enumitem,
    dsfont,
    marginnote,
}

% 2024-01-13 -------------------------
\usepackage[framemethod=default,usetwoside=false]{mdframed}

\mdfdefinestyle{theoremstyle}{%
    linecolor=black,
    linewidth=0.5pt,
    frametitlerule=true,
    frametitlerulewidth=0.5pt, 
    frametitlebackgroundcolor=white,
    frametitleaboveskip=4pt,
    frametitlebelowskip=3pt,
    frametitlefont=\nimbusfont,
    innerleftmargin=6pt,
    innerrightmargin=6pt,
}
\mdfsetup{skipabove=\topskip,skipbelow=0.5\topskip}
\mdtheorem[style=theoremstyle, theoremseparator={\hspace{7pt}}, theoremtitlefont=\itshape]
    {definition}{Definition}[chapter]
\mdtheorem[style=theoremstyle, theoremseparator={\hspace{7pt}}, theoremtitlefont=\itshape]
    {theorem}[definition]{Theorem}

\newtheoremstyle{uptheorem} 
{\baselineskip}{\baselineskip}{\upshape}{}{\bfseries}{.}{ }{} 
\theoremstyle{uptheorem}
\newtheorem{example}[definition]{Example}
\newtheorem{lemma}[definition]{Lemma}

\newtheoremstyle{uptheoremv3}{\baselineskip}{}{\upshape\small}{}{\bfseries}{.}{ }{} 
\theoremstyle{uptheoremv3}
\newtheorem{remark}[definition]{Remark}


% Custom font --------------------------------------------

\newfontfamily\nimbusfont{NimbusSanL.ttf}[
  Path = nimbussanl/,
  Extension = .ttf,
  UprightFont = *,  
  BoldFont = *-Bold,
  ItalicFont = *-Italic,
  BoldItalicFont = *-BoldItalic,
  Scale = 0.92,
]

\titleformat{\chapter}[display]
  {\nimbusfont\mdseries\Huge}
  {\huge\chaptertitlename\ \thechapter}
  {1em}
  {}

\titleformat{\section}
  {\nimbusfont\mdseries\huge}
  {\thesection}
  {1em}
  {}

\titleformat{\subsection}
  {\nimbusfont\mdseries\Large}
  {\thesubsection}
  {1em}
  {}

\titleformat{\subsubsection}
  {\nimbusfont\mdseries}
  {\thesubsubsection}
  {1em}
  {}

\captionsetup{
    labelfont=bf,
    font=footnotesize,
    justification=raggedright, % Left-align the caption
    singlelinecheck=false, % Disable single line check for multi-line captions
    format=plain,
    justification=justified,
}
% --------------------------------------------
% Layout & measures

\raggedbottom
\setlength{\parskip}{0.5\baselineskip}
\footmarkstyle{\textsuperscript{#1}\;}    % indent the footmark 
\setlength{\footmarkwidth} {0em}
\setlength{\footmarksep}   {0em}

\geometry{
    % showframe,
    margin=1.5cm,
    includeheadfoot,
    includemp,
    marginparsep=0.75cm,
    marginparwidth=5.25cm
}

\pagestyle{fancy}
\fancyhead{} % clear all header fields
\fancyhead[LE]{\nimbusfont\mdseries\nouppercase{\leftmark}}
\fancyhead[RO]{\nimbusfont\mdseries\nouppercase{\rightmark}}
\fancyfoot{} % clear all footer fields
\fancyfoot[LE,RO]{\thepage}
\renewcommand{\headrulewidth}{0pt}

\fancypagestyle{plain}{% % <-- this is new
  \fancyhf{} 
  \fancyfoot[LE,RO]{\thepage} % same placement as with page style "fancy"
  \renewcommand{\headrulewidth}{0pt}}

\footnotesinmargin

% SIDECAPTIONS --------------------------------------
\setsidecaps{\marginparsep}{\marginparwidth}
\sidecapmargin{outer}
\setsidecappos{t}
\renewcommand*{\sidecapstyle}{%
\captionnamefont{\bfseries\foottextfont}
\captionstyle{\RaggedRight\footnotesize\foottextfont}
}

% FULLWIDTH environment
% The following code should be used *after* any changes to the margins and
% page layout are made (e.g., after the geometry package has been loaded).
\newlength{\fullwidthlen}
\setlength{\fullwidthlen}{\marginparwidth}
\addtolength{\fullwidthlen}{\marginparsep}

\newenvironment{fullwidth}{%
  \begin{adjustwidth*}{}{-\fullwidthlen}%
}{%
  \end{adjustwidth*}%
}

% Custom commands --------------------------------------------

\newcommand{\qand}{\quad\text{and}\quad}
\newcommand{\chr}[1]{\textcolor{red}{#1}}
\newcommand{\C}{\mathbb{C}}
\newcommand{\R}{\mathbb{R}}
\newcommand{\Q}{\mathbb{Q}}
\newcommand{\Z}{\mathbb{Z}}
\newcommand{\N}{\mathbb{N}}
\renewcommand{\mathbb}[1]{\mathds{#1}}
\renewcommand{\subset}{\subseteq}

\DeclareMathOperator{\ob}{Ob}
\DeclareMathOperator{\ar}{Mo}
\DeclareMathOperator{\dom}{dom}
\DeclareMathOperator{\cod}{cod}



% --------------------------------------------

\usepackage{hyperref}
\hypersetup{
    bookmarksnumbered,
    hidelinks,
    colorlinks=true,
    allcolors=blue,
}

% --------------------------------------------
% Bibliography

\usepackage[style=numeric]{biblatex}
\addbibresource{bibliography.bib}

% 2024-01-18 --------------------------------
\renewcommand*{\raggedleftmarginnote}{}
\renewcommand*{\marginfont}{\footnotesize}
\newcommand{\doubt}[1]{\marginnote{\textcolor{red}{#1}}}
\newcommand{\note}[1]{\marginnote{#1}}
\newcommand{\red}[1]{\textcolor{red}{#1}}

% 2024-01-19 -----------------------------------
\setlist[itemize,enumerate]{topsep=0.25\baselineskip, itemsep=0.25\baselineskip}
% \setlist[itemize]{left=0pt}

% 2024-01-28 from Gilles ----------------------
\usepackage{import}
\usepackage{xifthen}
\usepackage{pdfpages}
\usepackage{transparent}
\newcommand{\incfig}[1]{%
    \def\svgwidth{\linewidth}
    \import{./images/}{#1.pdf_tex}
}


\newlength{\mylongest}
\setlength{\mylongest}{\widthof{\(\mathbf{\downdownarrows}\)}}
\setlength{\mylongestcat}{\widthof{\(\mathbf{\downdownarrows}\)}}
\addtolength{\mylongest}{\labelsep}
\SetLabelAlign{fincats}{\makebox[\mylongest]{#1}}
\SetLabelAlign{setcats}{\makebox[\mylongestcat]{#1}}


